\documentclass{article}

\usepackage[utf8]{inputenc}
\usepackage[margin=1.5cm]{geometry}
\usepackage[english]{babel}
\usepackage{array,xcolor}
\usepackage{hyperref}
\usepackage{multicol}

\definecolor{lightgray}{gray}{0.8}
\newcolumntype{L}{>{\raggedleft}p{0.20\textwidth}}
\newcolumntype{R}{p{0.80\textwidth}}
\newcommand\VRule{\color{lightgray}\vrule width 0.5pt}

\setlength{\parindent}{0pt}
\pagenumbering{gobble}

\title{\vspace{-1.5em}Jacob Thomas Errington}
\author{\texttt{jake@mail.jerrington.me} \\ $+1\;(514)\;503-3100$}
\date{}

\begin{document}

\maketitle

\hrule

\subsection*{Work experience}

\begin{tabular}[h]{L!{\VRule}R}
    Jan. 2015 to Sept. 2015
        & CTO \& Cofounder, full-stack web development, native Android development               \\
        & Ahvoda Recruitment Inc., \url{http://ahvoda.com/}                                      \\
        & Participated in McGill's X-1 startup accelerator, \url{http://mcgillx1accelerator.com} \\
    Jun. 2014 to Jun. 2015
        & Research assistant, scientific programming in genetics, big data \\
        & McGill University and Genome Quebec Innovation Centre            \\
        & 740 Avenue du Dr Penfield, Montreal, Quebec                      \\
        & Supervised by \href{http://simongravel.lab.mcgill.ca/Home.html}{Simon Gravel}, \texttt{simon.gravel@mcgill.ca}
\end{tabular}

\subsection*{Education}

\begin{tabular}[h]{L!{\VRule}R}
    2014-2017 & B.Sc. Mathematics \& Computer Science, McGill University, Montreal, Quebec   \\
              & CGPA $3.62$                                                                  \\
    2012-2014 & D\'EC in Pure and Applied Science (Honours), Dawson College, Westmount.      \\
              & Admitted to the Honour Roll every semester; Final overall average above 90\%.
\end{tabular}

\subsection*{Volunteer \& Outreach}

\begin{tabular}[h]{L!{\VRule}R}
    Sept. 2015 to \emph{present}
        & Outreach director at HackMcGill, McGill University's hacker club,
            \url{http://hackmcgill.com} \\
        & Organize community coding events and tutorials, participate in organizing McHacks
\end{tabular}

\subsection*{Projects}

\begin{description}
    \item[CBKB.] In 36 hours at MHacks Fall 2015, my team and I designed an
        all-emoji, esoteric functional programming language and implemented
        a compiler for it targeting JavaScript.

        \url{https://github.com/labcoders/crystalball-knife-bomb}

    \item[HackSignal.] At hackathons, it can be hard to find the mentors that
        you need when you get stuck. HackSignal is an online platform for peer
        mentorship at hackathons that emphasizes real-life interaction between
        the mentor and mentee. A rewrite of the webapp into Haskell from Python
        is in progress.

        \url{https://github.com/djeik/hacksignal}~
        \url{https://github.com/djeik/hacksignal-naph}

    \item[FishBucketChallenge.] To help scientists working on medication for
        ALS (Lou Gehrig's disease), in 24 hours my team developed computer
        vision software in Python to track the movement of zebra fish, and
        determine their maximum velocity based only on low-resolution grayscale
        videos of the fish.

        \url{https://github.com/ndejay/fish-bucket-challenge}

    \item[Text4coffee.] Use a Raspberry Pi and the Twilio API to turn a
        run-of-the-mill coffee machine into an Internet of things brewing
        station. Built with Flask and Postgres at Wearhacks Montreal 2015.

        \url{https://github.com/labcoders/text4coffee}

    \item[Vent/Tea.] Online platform to match people who need someone to talk
        to with trained listeners. The pairs of people meet up physically over
        tea or coffee to talk, rather than simply discuss through an online
        chat.

        \url{https://github.com/vent-over-tea/vent-over-tea}~
        \url{http://ventovertea.com/}

    \item[TeXnote.] Web application for editing and previewing \LaTeX{} documents
        directly in the browser. Images can be easily inserted into documents
        via email thanks to the SendGrid API. Built with Node.js and
        socket.io at MHacks Winter 2015.
        \emph{Winner of the SendGrid API prize.}

        \url{https://github.com/labcoders/texnote}
\end{description}

\end{document}
