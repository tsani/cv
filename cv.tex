\documentclass{article}

\usepackage[utf8]{inputenc}
\usepackage[margin=1.5cm]{geometry}
\usepackage[english]{babel}
\usepackage{array,xcolor}
\usepackage{hyperref}
\usepackage{multicol}

\definecolor{lightgray}{gray}{0.8}
\newcolumntype{L}{>{\raggedleft}p{0.20\textwidth}}
\newcolumntype{R}{p{0.80\textwidth}}
\newcommand\VRule{\color{lightgray}\vrule width 0.5pt}

\setlength{\parindent}{0pt}
\pagenumbering{gobble}

\title{\vspace{-1.5em}Jacob Thomas Errington}
\author{\texttt{jake@mail.jerrington.me} -- \url{https://jerrington.me} \\ $+1\;(514)\;503-3100$}
\date{}

\begin{document}

\maketitle

\hrule

\subsection*{Work experience}

\begin{tabular}[h]{L!{\VRule}R}
  Jun. 2017 to Aug. 2017
    & Software developer -- 10-week internship program \\
    & 1010data Services Inc., \url{http://1010data.com} \\
    & Linux systems programming in C, application programming in
      \href{https://en.wikipedia.org/wiki/K_\%28programming_language\%29}{K}.
      \\
    & Improved login time by a factor of 12. \\
    & Developed new infrastructure for debugging K applications. \\
    & \\
  Feb. 2016 to Feb. 2017
    & Software engineer -- lead integration developer \\
    & OOHLALA Mobile Inc., \url{http://oohlalamobile.com} \\
    & Backend web and server-side development; Python, SQL. \\
    & Developed integrations with existing institutional infrastructure. \\
    & \\
  Jan. 2015 to Sept. 2015
    & CTO and cofounder \\
    & Ahvoda Recruitment, Inc. (now closed) \\
    & Participated in McGill's X-1 startup accelerator,
      \url{http://mcgillx1accelerator.com} \\
    & Full-stack web development, native Android development,
      database administration \\
    & \\
  Jun. 2014 to Jun. 2015
    & Research assistant, scientific programming in genetics, big data. \\
    & McGill University and Genome Quebec Innovation Centre \\
    & Supervised by
      \href{http://simongravel.lab.mcgill.ca/Home.html}{Simon Gravel},
      \href{mailto:simon.gravel@mcgill.ca}{simon.gravel@mcgill.ca}
\end{tabular}

\subsection*{Education}

\begin{tabular}[h]{L!{\VRule}R}
  2014 to 2017
    & B.Sc. Mathematics \& Computer Science, McGill University,
      Montreal, Quebec \\
    & CGPA $3.67$ \\
    & Relevant courses: compiler design, computability theory,
      logic and computation, set theory. \\
  2012 to 2014
    & D\'EC in Pure and Applied Science (Honours),
      Dawson College, Westmount. \\
    & Admitted to the Honour Roll every semester.
      Final overall average above 90\%.
\end{tabular}

\subsection*{Volunteer \& Outreach}

\begin{tabular}[h]{L!{\VRule}R}
  Sept. 2015 to May 2016
    & Outreach director at HackMcGill, McGill University's hacker club. \\
    & Organized community events and tutorials,
      participated in organizing McHacks. \\
    & \url{http://mchacks.io/}.
\end{tabular}

\subsection*{Projects}

\begin{description}
  \item[Fast accumulator startup.]
    In a 10-week internship at 1010data, I was tasked with speeding up the
    core platform login. Logging in involves starting up an accumulator
    process, written in K, to execute the user's requests. My work
    accelerates accumulator startup by introducing a caching mechanism for
    accumulators based on forking: partially started accumulators are
    cached, which allows skipping the loading procedure for new sessions.

  \item[Goto.]
     For a course on compiler design, my partner and I implemented a compiler
     for a subset of Go called GoLite in Haskell. Our compiler featured a
     partial x86\_64 code generator, as well as a complete C++ code generator.
     Our C++ generator emitted faster code than the reference compiler.

     For outstanding achievement in the course, I was personally emailed by the
     professor -- Dr. Laurie Hendren, ACM fellow, Canada Research Chair, and
     fellow of the Royal Society of Canada -- to congratulate me.

  \item[servant-github-webhook.] A Haskell library that provides easy to
    use combinators for the \texttt{servant} web framework for handling
    GitHub webhooks.

    \url{https://github.com/tsani/servant-github-webhook}

  % \item[CBKB.] In 36 hours at MHacks Fall 2015, my team and I designed an
  %     all-emoji, esoteric functional programming language and implemented
  %     a compiler for it targeting JavaScript.

  %     \url{https://github.com/labcoders/crystalball-knife-bomb}

  \item[FishBucketChallenge.]
    To help scientists working on medication for ALS (Lou Gehrig's disease), in
    24 hours my team developed computer vision software in Python to track the
    movement of zebra fish, and determine their maximum velocity based only on
    low-resolution grayscale videos of the fish.

    \url{https://github.com/ndejay/fish-bucket-challenge}

  % \item[Text4coffee.]
  %   Use a Raspberry Pi and the Twilio API to turn a
  %   run-of-the-mill coffee machine into an Internet of things brewing
  %   station. Built with Flask and Postgres at Wearhacks Montreal 2015.

  %   \url{https://github.com/labcoders/text4coffee}

  % \item[Vent over Tea.] Online platform to match people who need someone to
  %   talk to with trained listeners. The pairs of people meet up physically
  %   over tea or coffee to talk, rather than simply discuss through an
  %   online chat.

  %   Growing the project after I left, the two founders of Vent over Tea
  %   went on to win the first place prize of \$10~000 in the McGill
  %   University Dobson Cup Startup Contest.

  %   \url{https://github.com/vent-over-tea/vent-over-tea}~
  %   \url{http://ventovertea.com/}

  \item[TeXnote.]
    Web application for editing and previewing \LaTeX{} documents directly in
    the browser. Images can be easily inserted into documents via email thanks
    to the SendGrid API. Built with Node.js and socket.io at MHacks Winter
    2015. \emph{Winner of the SendGrid API prize.}

    \url{https://github.com/labcoders/texnote}
\end{description}

\end{document}
